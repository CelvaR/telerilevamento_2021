% define text size and document class
\documentclass[9pt]{beamer}
\usetheme{Singapore}
\usefonttheme{structuresmallcapsserif}
\usecolortheme{dove}
% call packages
\usepackage[utf8]{inputenc}
\usepackage{listings} % needed to import code strings
\usepackage[export]{adjustbox}
\usepackage{graphicx}
\usepackage{color}
\usepackage{tikz}
\usepackage{multicol} % to put equations side by side
\usepackage{natbib}
\usepackage{ragged2e}

% set images size
\pgfdeclareimage[width=\paperwidth,height=\paperheight]{mybackground}{IMG_3656c.JPG}
\pgfdeclareimage[width=0.3\textwidth]{logo}{LogoUNIBO.png}

\setbeamertemplate{title page}{
	
	\begin{picture}(0,0)
		\put(-32,-164){ % \put(xcoord, ycoord)
			\pgfuseimage{mybackground}
		}
	
		% \put(230,30){%
		\put(100, -150){
		\pgfuseimage{logo}
		}
		
		\put(-25, 5){%
			\begin{minipage}{\paperwidth}
				\begin{center}
				\usebeamerfont{title}{\inserttitle\par}
				\smallskip
				\usebeamerfont{author}{\insertauthor\par}
				\medskip
				\usebeamerfont{author}{\insertinstitute\par}
				\end{center}
			\end{minipage}
		}

	\end{picture}
}

\title[...]{\LARGE Storm "Vaia" effects on a local scale}
\author{\normalsize Roberto Celva \inst{1}}
\institute{\small \inst{1} Environmental Assessment and Management
% "\\" to start a new line. alternatives are "\bigskip" and "\smallskip"
}

\begin{document}
	
	\begin{frame}[plain]
		\titlepage
	\end{frame}
	
	\section{1. Introduction}
	\subsection{1.1. Overview}
		\begin{frame}{Overview}
				\justifying
				Between the $27^{th}$ and the $30^{th}$ of October, 2018, an atmosferic depression generated in northern Europe triggered the upsurge of mediterranean air.\\
				
				\bigskip
				In eastern Italian Alps, winds up to 200km/h caused extensive damages to the alpine ecosystem, affecting 3\% up to 47\% of local forests \citep{chirici}.
			\begin{figure}
				\centering
				\includegraphics[width=.6\paperwidth, height=.4\paperheight]{IMG_3658.JPG} \caption{\scriptsize Aftermath of Vaia storm in Lavazè area, Fiemme valley}
			\end{figure}
		\end{frame}
	
	\subsection{1.2. Study area}
	\begin{frame}{Study area}
		\frametitle{Study area}
		\justifying
		Fiemme valley was heavily hit by the storm. Forest management mainly falls under Cavalese forest district, which is set as the main area of interest.
	\begin{figure}
		\includegraphics[width=0.4\paperwidth]{Layout2.pdf}
		\includegraphics[width=0.4\paperwidth]{Layout1.pdf}
		\caption{\scriptsize Location of the study area. The spatial polygon of Cavalese forest district (right) was obtained from \textcolor{red}{Trento Province Geocatalog} \href{https://siat.provincia.tn.it/geonetwork}{\beamergotobutton{Link}}}
	\end{figure}
\end{frame}

\section{2. Methods}
\begin{frame}{Outlook}
	\justifying
	Forest damages was assessed using Landsat8 data (path 192, row 028) collected during Oct. $4^{th}, 2018$ and Sept. $21^{st}, 2019$ repeats.
	
	\bigskip
	Raster files were downloaded using \textcolor{red}{EarthExplorer data portal} \href{https://earthexplorer.usgs.gov/}{\beamergotobutton{Link}}\\
\end{frame}
	
\section{3. Results}
	\subsection{3.1. RGB colors}
		\begin{frame}{RGB colors}
			\includegraphics[width=\textwidth]{Rplot.pdf}
			\lstinputlisting[language=R, basicstyle=\scriptsize]{Rplot.r}
		\end{frame}
	\subsection{3.2. NIR enhancement}
		\begin{frame}{NIR enhancement}
			\includegraphics[width=\textwidth]{Rplot01.pdf}
			\lstinputlisting[language=R, basicstyle=\scriptsize]{Rplot01.r}
		\end{frame}
	
	% [trim={<left> <lower> <right> <upper>},clip] crops margins in bp (BIG POINTS) units
	\subsection{3.3. Shift in RED}
		\begin{frame}{Shift in RED}
			\centering
			\includegraphics[width=0.4\paperwidth, trim={0 100 0 120},clip]{Rplot04.pdf}
			\lstinputlisting[language=R, basicstyle=\tiny]{Rplot034.r}
		\end{frame}

\section{4. Discussion}
\subsection{4.1 Lost forest}
	\begin{frame}{Lost forest}
	\begin{picture}(0,0)
			\put(200,-20){%
		\begin{minipage}{0.5\paperwidth}
			\includegraphics[width=0.3\paperwidth]{Rplot05.pdf}
		\end{minipage}
	}
			\put(0,-10){%
		\begin{minipage}{0.5\paperwidth}
			\lstinputlisting[language=R, basicstyle=\tiny]{Rplot05.r}
		\end{minipage}
	}
	\end{picture}
	\end{frame}

\subsection{4.2 nDVI}
	\begin{frame}{nDVI}
		\begin{multicols}{2}
			\begin{equation}
				\scriptsize nDVI_{t=i} = \frac{DVI_{t=i}}{(RED + NIR)_{t=i}} \break
			\end{equation} \\
			\begin{equation}
				\scriptsize \Delta nDVI = nDVI_{t=0} - nDVI_{t=1}
			\end{equation}
		\end{multicols}
	
		\begin{picture}(0,150)
			\put(0,90){%
				\begin{minipage}{0.5\paperwidth}
					\lstinputlisting[language=R, basicstyle=\tiny]{Rplot06.r}
				\end{minipage}
			}
			\put(180,0){%
				\includegraphics[width=0.4\paperwidth, trim={0 70 0 120},clip]{Rplot06.pdf}
			}
		
		\end{picture}
	
	\end{frame}

\section{5. Conclusions}
	\begin{frame}{Conclusions}
		\begin{picture}(0,0)
 			\put(0,-20){%
 			\begin{minipage}{0.5\paperwidth}
 				\includegraphics[width=0.2\paperwidth, height=0.3\paperheight]{Rplot07.pdf}
 				\includegraphics[width=0.2\paperwidth, height=0.3\paperheight]{Rplot08.pdf}\\
 				\includegraphics[width=0.2\paperwidth, height=0.3\paperheight,trim={25 30 90 30},clip]{Rplot09.pdf}
 				\includegraphics[width=0.2\paperwidth, height=0.3\paperheight,trim={25 30 90 30},clip]{Rplot10.pdf}
 			\end{minipage}
 		}
 			\put(170,-20){%
 			\parbox{\textwidth}{
 			Unsupervised classification has\\
 			performed poorely, probably\\
 			influenced by unhomogeneous\\
 			reflectance values.\\
 		
 			This may be due in part to\\
 			partial snow cover (2019),\\
 			in part to difference in sunlight\\
 			incidence on mountaneous surfaces\\
 			at different time of the day.\\
 		
	 		An example is provided by\\
 			the figures to the left.\\ 
 			}
 		}
		\end{picture}
	\end{frame}

	\begin{frame}{References}
		\begin{thebibliography}{100}
		\bibitem[Chirici et al., 2019]{chirici}
		\scriptsize Chirici G, Giannetti F, Travaglini D, Nocentini S, Francini S, D’Amico G, Calvo E, Fasolini D, Broll M, Maistrelli F, Tonner J, 
		Pietrogiovanna M, Oberlechner K, Andriolo A, Comino R, Faidiga A, Pasutto I, Carraro G, Zen S, Contarin F, Alfonsi L, Wolynski A, Zanin M, 
		Gagliano C, Tonolli S, Zoanetti R, Tonetti R, Cavalli R, Lingua E, Pirotti F, Grigolato S, Bellingeri D, Zini E, Gianelle D, Dalponte M, Pompei 
		E, Stefani A, Motta R, Morresi D, Garbarino M, Alberti G, Valdevit F, Tomelleri E, Torresani M, Tonon G, Marchi M, Corona P, Marchetti M 
		(2019). Stima dei danni della tempesta “Vaia” alle foreste in Italia. Forest@ 16: 3-9. – doi: 10.3832/efor3070-016
		\end{thebibliography}
	
		\bigskip
		\bigskip
		\centering
		\underline {\scriptsize Thank you for the attention!}
	\end{frame}
\end{document}